\makeatletter
\usetikzlibrary{circuits.ee.IEC}
\def\tikzDipoleWidth{2cm}

\tikzset{%
  circuit declare symbol=dummy dipole,
  set dummy dipole graphic={%
    draw,
    shape=cloud,
    minimum width=1cm,
    minimum height=.5cm,
    transform shape}}
\pgfdeclareshape{ground IEC}
{
  \inheritsavedanchors[from=rectangle ee]
  \inheritanchor[from=rectangle ee]{center}
  \inheritanchor[from=rectangle ee]{north}
  \inheritanchor[from=rectangle ee]{south}
  \inheritanchor[from=rectangle ee]{east}
  \inheritanchor[from=rectangle ee]{west}
  \inheritanchor[from=rectangle ee]{north east}
  \inheritanchor[from=rectangle ee]{north west}
  \inheritanchor[from=rectangle ee]{south east}
  \inheritanchor[from=rectangle ee]{south west}
  \inheritanchor[from=rectangle ee]{input}
  \inheritanchor[from=rectangle ee]{output}
  \inheritanchorborder[from=rectangle ee]

  \backgroundpath{%
    \pgf@process{%
      \pgfpointadd{\southwest}{%
        \pgfpoint{\pgfkeysvalueof{/pgf/outer xsep}}%
        {\pgfkeysvalueof{/pgf/outer ysep}}}}
    \pgf@xa=\pgf@x
    \pgf@ya=\pgf@y
    \pgf@process{%
      \pgfpointadd{\northeast}{%
        \pgfpointscale{-1}{%
          \pgfpoint{\pgfkeysvalueof{/pgf/outer xsep}}%
          {\pgfkeysvalueof{/pgf/outer ysep}}}}}
    \pgf@xb=\pgf@x
    \pgf@yb=\pgf@y
    % On s'abaisse un peu
    \pgf@xc=.5\pgf@xb
    \pgf@yc=\pgf@ya
    \advance\pgf@yc\pgf@yb
    \pgfpathmoveto{\pgfqpoint{\pgf@xa}{\pgf@yc}}
    \pgfpathlineto{\pgfqpoint{\pgf@xc}{\pgf@yc}}
    % First line, simple
    \pgfpathmoveto{\pgfqpoint{\pgf@xc}{\pgf@ya}}
    \pgfpathlineto{\pgfqpoint{\pgf@xc}{\pgf@yb}}
    % Hachure
    \pgfmathsetlength\pgfutil@tempdima{(\pgf@yb-\pgf@ya)*.1}
    % \pgf@xa reste constant
    % \pgf@xb reste constant et égal à \pgf@xa + longueur du trait
    \pgfmathsetlength\pgf@xb{\pgf@xc+.1cm}
    \pgfmathloop\ifnum\pgfmathcounter<12\relax
      \pgfmathsetlength\pgf@yb{\pgf@ya+.5\pgfutil@tempdima}
      \pgfpathmoveto{\pgfqpoint{\pgf@xc}{\pgf@ya}}
      \pgfpathlineto{\pgfqpoint{\pgf@xb}{\pgf@yb}}
      \pgfmathsetlength\pgf@ya{\pgf@ya+\pgfutil@tempdima}
    \repeatpgfmathloop
  }
}
\makeatother

%%% Local Variables:
%%% mode: latex
%%% TeX-master: t
%%% End:
