%Fichier contenant les raccourcis % 
%%%%%%%%%%%%%%%%%%%%%%%%%%%%%%%%%%%
%\renewcommand{\phi}{\ensuremath{\varphi}}

%\renewcommand{\min}{\ensuremath{\mathrm{min}}}
%\newcommand{\max}{\ensuremath{\mathrm{max}}}

\newcommand{\bs}{\boldsymbol} 
\newcommand{\mbf}[1]{\ensuremath{\mathbf{#1}}}

\newcommand{\vect}[1]{\ensuremath{\vv{#1}}\xspace}
\newcommand{\vecti}[2]{\ensuremath{\vv*{#1}{#2}}\xspace}

\newcommand{\mcal}{\ensuremath{\mathcal}}

\newcommand{\R}{\ensuremath{\mcal{R}}\xspace}
\newcommand{\Rg}{\ensuremath{\mcal{R}_\text{g}}\xspace}
\newcommand{\RG}{\ensuremath{\mcal{R}_\text{G}}\xspace}
\newcommand{\Rt}{\ensuremath{\mcal{R}_\text{T}}\xspace}
\newcommand{\aT}{\vecti{a}{\Rt}}
\newcommand{\vR}{\vecti{v}{\R}}
\newcommand{\pR}{\vecti{p}{\R}}
\newcommand{\vit}{\vect{v}}
\newcommand{\qdm}{\vect{p}}
\newcommand{\qdmg}{\vecti{p}{\Rg}}
\newcommand{\ag}{\vecti{a}{\Rg}}
\newcommand{\aR}{\vecti{a}{\R}}

%\renewcommand{\d}{\ensuremath{\mrm{d}}}
\renewcommand{\d}{\D}



\newcommand{\cyc}{\ensuremath{\circlearrowleft}}

\newcommand{\mrm}{\ensuremath{\mathrm}}

\newcommand{\bv}[1]{\ensuremath{{\bf #1}}}

\newcommand{\e}[1]{\ensuremath{\vv*{e}{#1}}\xspace}
\newcommand{\ex}{\e{x}}
\newcommand{\ey}{\e{y}}
\newcommand{\ez}{\e{z}}
\newcommand{\ei}{\e{i}}
\newcommand{\ej}{\e{j}}

\newcommand{\er}{\e{r}}
\newcommand{\et}{\e{\theta}}
\newcommand{\ep}{\e{\phi}}


\newcommand{\eq}{\ensuremath{\mrm{eq}}}

\newcommand{\dt}[1]{\ensuremath{\mathop{\diff{#1}{t}}\nolimits\xspace}}
\newcommand{\dtheta}[1]{\ensuremath{\diff{#1}{\theta}}\xspace}
\newcommand{\dx}[2]{\ensuremath{\diff{#1}{#2}}\xspace}
\newcommand{\ddx}[2]{\ensuremath{\diff[2]{#1}{{#2}}}\xspace}
\newcommand{\dnx}[3]{\ensuremath{\diff[#3]{#1}{#2}}\xspace}
\newcommand{\ddt}[1]{\ensuremath{\diff[2]{#1}{t}}\xspace}
\newcommand{\ddtheta}[1]{\ensuremath{\diff{#1}{\theta^2}}\space}

% On ne peut pas utiliser \diffp, il y a un conflit avec mpsi.cls
\newcommand{\drondx}[3][1]{\ensuremath{\paren{\frac{\partial #2}{\partial #3}}_{#1}}\xspace}
\newcommand{\ddrondx}[3][1]{\ensuremath{\paren{\frac{\partial^2 #2}{\partial {#3}^2}}_{#1}}\xspace}
\newcommand{\ddrondxy}[4][1]{\ensuremath{\paren{\frac{\partial^2
        #2}{\partial #3 \partial #4}}_{#1}}\xspace}

\newcommand{\uR}{\ensuremath{u_\mrm{R}}}
\newcommand{\uC}{\ensuremath{u_\mrm{C}}}
\newcommand{\uL}{\ensuremath{u_\mrm{L}}}
\newcommand{\UR}{\ensuremath{U_\mrm{R}}}
\newcommand{\UC}{\ensuremath{U_\mrm{C}}}
\newcommand{\UL}{\ensuremath{U_\mrm{L}}}
\newcommand{\half}{\ensuremath{\frac{1}{2}}}
\newcommand{\inv}[1]{\ensuremath{\frac{1}{#1}}}
\newcommand{\diffx}{\mrm{d}x}
\newcommand{\diffy}{\mrm{d}y}
\newcommand{\diffz}{\mrm{d}z}
\newcommand{\bra}{\langle}
\newcommand{\ket}{\rangle}


\newcommand{\legende}[1]{\small\textsf{#1}} % l�gendes hors des flottants

\newcommand{\li}{$^7$Li} 
\newcommand{\lix}{$^6$Li}
\newcommand{\lixd}{$^6$Li$_2$}
\newcommand{\kb}{\ensuremath{k_\mrm{B}}} 
\newcommand{\Na}{\ensuremath{\mcal{N}_\mrm{A}}}
\newcommand{\ldb}{\lambda_\mrm{dB}}
\newcommand{\mub}{\mu_\mathrm{B}} 
\newcommand{\fket}[2]{$\vert F=#1,m_F=#2\rangle$} 
\newcommand{\Ket}[1]{$\vert{#1}\rangle$}
\newcommand{\textKet}[1]{\vert{#1}\rangle}
\newcommand{\Bra}[1]{$\langle{#1}\vert$}
\newcommand{\textBra}[1]{\langle{#1}\vert}
\newcommand{\textBraKet}[2]{\langle{#1}\vert{#2}\rangle}
\newcommand{\omegarad}{\omega_{\perp}}
\newcommand{\sigmarad}{\sigma_{\perp}}
\newcommand{\omegapar}{\omega_{\vert\vert}}
\newcommand{\sigmapar}{\sigma_{\vert\vert}}
\newcommand{\omegabar}{\bar{\omega}}
\newcommand{\sigmabar}{\bar{\sigma}} 
\newcommand{\ef}{E_{\mathrm F}}
\newcommand{\kf}{k_{\mathrm F}} 
\newcommand{\tf}{T_{\mathrm F}}
\newcommand{\alphas}{\alpha_\mathrm{S}} 
\newcommand{\somm}{_\mathrm{S}}
\newcommand{\mean}[1]{\langle#1\rangle}
\newcommand{\commut}[2]{\left[#1,#2\right]}
\newcommand{\paren}[1]{\left(#1\right)}
\newcommand{\eint}{\protect{\ensuremath{\mcal{E}_\mrm{int}}}\xspace} 
\newcommand{\epint}{\protect{\ensuremath{\mcal{E}_\mrm{p,int}}}\xspace}
\newcommand{\epext}{\protect{\ensuremath{\mcal{E}_\mrm{p,ext}}}\xspace}
\newcommand{\emec}{\protect{\ensuremath{\mcal{E}_\text{m}}}\xspace} 
\newcommand{\ekin}{\protect{\ensuremath{\mcal{E}_\text{cin}}}\xspace}
\newcommand{\epot}{\protect{\ensuremath{\mcal{E}_\mrm{pot}}}\xspace}
\newcommand{\eeff}{\protect{\ensuremath{\mcal{E}_\mrm{eff}}}\xspace}
\newcommand{\erel}{\mcal{E}_\mrm{lib}}
\newcommand{\etot}{\mcal{E}_\mrm{tot}} 
\newcommand{\tc}{T_\mrm{C}}
\newcommand{\tbec}{T_\mrm{CBE}}
\newcommand{\pinch}{\og pinch \fg}

\newcommand{\cfbox}[1]{\begin{center}\fbox{#1}\end{center}}
\newcommand{\boite}[1]{\fbox{$\displaystyle #1$}}
\newcommand{\whiteboite}[1]{\fbox{$\displaystyle\textcolor{white}{\LARGE
      #1}$}}
\newcommand{\ie}{{\it ie}\xspace}
\newcommand{\opt}[1]{\frac{1}{\overline{#1}}}
%\newcommand{\vect}{\overrightarrow}
\newcommand{\ul}[1]{\underline{#1}}
\newcommand{\ol}[1]{\overline{#1}}
\newcommand{\defi}{\equiv}


\protected\def\Epot{\@ifnextchar[{\Epot@ii}{\Epot@i}}
\def\Epot@i{\Energie_{\text{pot}}}
\def\Epot@ii[#1]{\Energie_{\text{pot},#1}}

\newcommand{\Ggrav}{\mathcal{G}}

\protected\def\DerivRef{%
  \@ifnextchar*{\DerivRef@i}{\DerivRef@ii}}
\def\DerivRef@i*#1#2#3{%
  \Deriv*{#1}{#2}{\Referentiel{#3}}}
\def\DerivRef@ii#1#2{%
  \Deriv*{#1}{t}{\Referentiel{#2}}}

% \protected\def\DerivRefDeux{%
%   \@ifnextchar*{\DerivRefDeux@i}{\DerivRefDeux@ii}}
% \def\DerivRefDeux@i*#1#2#3{%
%   \Deriv*[2]{#1}{{#2}{\Referentiel{#3}}}}
% \def\DerivRefDeux@ii#1#2{%
%   \Deriv*[2]{#1}{t}{\Referentiel{#2}}}

% \let\Delem\Diff

% \protected\def\Puiss{%
%   \@ifnextchar*{\Puiss@i}{\Puiss@ii}}
% \def\Puiss@i*#1#2{%
%   \mathcal{P}_{#2}\left(#1\right)}
% \def\Puiss@ii#1{%
%   \mathcal{P}\left(#1\right)}
% \protected\def\Trajet#1#2#3{
%   #1 \xrightarrow[#3]{}#2}

% \protected\def\Trav{%
%   \@ifnextchar*{\Trav@i}{\Trav@ii}}
% \def\Trav@ii{\@ifnextchar[{\Trav@j}{\Trav@jj}}
% \def\Trav@j[#1]#2{W_#1\left(#2\right)}
% \def\Trav@jj#1{W\left(#1\right)}
% \def\Trav@i*{\@ifnextchar[{\Trav@k}{\Trav@kk}}
% \def\Trav@k[#1]#2#3{\underset{#3}{W_#1}\left(#2\right)}
% \def\Trav@kk#1#2{\underset{#2}{W}\left(#1\right)}


\def\MesureAlgebrique#1{\overline{#1}}

% %% Tikz

% Le code suivant buggue dans beamer (peut �tre les environnements columns)
% Pour rendre inactifs les caract�res : et autres
% \AtBeginDocument{%
%   \@ifpackageloaded{babel}{%
%     \let\xCJtikz@tikzpicture\tikzpicture
%     \def\tikzpicture{%
%       \iflanguage{french}{%
%         \shorthandoff{:;?!}}{}%
%       \xCJtikz@tikzpicture}%
%     \let\xCJtikz@tikzexternal@tikzpicture@replacement%
%         \tikzexternal@tikzpicture@replacement
%     \def\tikzexternal@tikzpicture@replacement{%
%       \iflanguage{french}{%
%         \shorthandoff{:;?!}}{}%
%       \xCJtikz@tikzexternal@tikzpicture@replacement}}{}}


\pgfplotsset{%
  % Thanks to @Jake
  SchoolPlot/.style = {%
    axis lines = middle,
    enlargelimits = true,
    %% La suite bug dans l'environnement mpsi_standalone
    % before end axis/.code = {
    %   % \addplot [draw=none, forget plot] coordinates {(0,0)};
    %   % \node at (axis cs:0,0) [anchor=north east] {0};
    % },
    xlabel style = {
      anchor = 130
    },
    ylabel style = {
      anchor = east
    },
  },
  SchoolPlot/.default = north east,
}

\pgfplotsset{%
  petit graphe/.style = {%
    width = 7cm},
  tres petit graphe/.style = {%
    width = 6cm},
}    

\pgfplotsset{%
  every axis legend/.append style ={fill = none}
}

\pgfplotsset{%
  compat = newest,
  cycle list name = exotic,
  scaled ticks = false,
  points de mesure/.style = {%
    every axis plot post/.append style = {%
      mark = +,
      only marks,
    }
  },
  gnuplot writes logscale = false,
  every axis plot/.append style = {thick},
  % every axis plot post/.append style = {mark = none},
  every axis/.append style = {%
    thick,
    enlargelimits = true,
    % minor tick num = 4,
  },
  every tick/.style = {very thick},
  every minor tick/.style = {%
    thick,
    gray},
  every colorbar/.append style = {%
    yticklabel pos = right,
    axis lines = box,
    every inner x axis line/.append style = {-},
    every inner y axis line/.append style = {-},
    every outer x axis line/.append style = {-},
    every outer y axis line/.append style = {-}},
}

% Author: Christophe Jorssen
% Public domain
\pgfkeysdef{/tikz/mark angle/start angle}{\tikzset{start angle=#1}}
\pgfkeysdef{/tikz/mark angle/end angle}{\tikzset{end angle=#1}}
\pgfkeysdef{/tikz/mark angle/angle radius}{\tikzset{radius=#1}}
\pgfkeyssetvalue{/tikz/mark angle/label radius}{1cm}
\pgfkeyssetvalue{/tikz/mark angle/label pos}{.5}
\pgfkeyssetvalue{/tikz/mark angle/node options}{}
\pgfkeyssetvalue{/tikz/mark angle/path options}{}
\def\tikzMarkAngle{%
  \pgfutil@ifnextchar[{\tikzMarkAngle@i}{\tikzMarkAngle@i[]}}
\def\tikzMarkAngle@i[#1](#2)(#3)(#4)#5{%
  % #1 optional parameters
  % #2 coordinate of the center
  % #3 coordinate giving the start direction
  % #4 coordinate giving the end direction
  % #5 label
  \bgroup
    \coordinate (xCJtikz@AngleCenter) at (#2);
    \coordinate (xCJtikz@AngleStart) at (#3);
    \coordinate (xCJtikz@AngleEnd) at (#4);
    \pgfmathanglebetweenpoints{%
      \pgfpointanchor{xCJtikz@AngleCenter}{center}}{%
      \pgfpointanchor{xCJtikz@AngleStart}{center}}
    \edef\AngleStart{\pgfmathresult}%
    \pgfmathanglebetweenpoints{%
      \pgfpointanchor{xCJtikz@AngleCenter}{center}}{%
      \pgfpointanchor{xCJtikz@AngleEnd}{center}}
    \edef\AngleEnd{\pgfmathresult}%
    \ifdim\AngleEnd pt<\AngleStart pt\relax
      \pgfmathsetmacro\AngleEnd{\AngleEnd+360}
    \fi
    \pgfkeys{%
      /tikz/mark angle/.cd,
      angle radius=1cm,
      label radius=1.2cm,
      label pos=.5,
      start angle=\AngleStart,
      end angle=\AngleEnd,
      #1}
      \edef\xCJ@temp{%
        \noexpand\draw[\pgfkeysvalueof{/tikz/mark angle/path options}]
        (\noexpand$(xCJtikz@AngleCenter)!\pgfkeysvalueof{/tikz/x
          radius}!(xCJtikz@AngleStart)\noexpand$) arc;
        \noexpand\node[\pgfkeysvalueof{/tikz/mark angle/node options}] at
        (\noexpand$(xCJtikz@AngleCenter)+(\AngleStart+\pgfkeysvalueof{/tikz/mark
          angle/label pos}*\AngleEnd-\pgfkeysvalueof{/tikz/mark
          angle/label pos}*\AngleStart:\pgfkeysvalueof{/tikz/mark angle/label radius})\noexpand$)}%
    \xCJ@temp{#5};%
  \egroup
  \ignorespaces}
% Local Variables:
% coding: utf-8
% End:


\tikzset{%
  Axis/.style={->,very thick},
  AxisPlot/.style={->,thick},
  Point/.style = {%
    fill = DarkDuotone,
    circle,
    minimum width=2mm,
    inner sep=0pt},
  3D/.style = {%
    x = {(-.5cm,-.5cm)},
    y = {(1cm,0cm)},
    z = {(0cm,1cm)}},
}

%% Tikz fin

%---- Ensembles : entiers, reels, complexes... ----
\newcommand{\Nn}{\mathbb{N}}
\newcommand{\Zz}{\mathbb{Z}}
\newcommand{\Qq}{\mathbb{Q}}
\newcommand{\Rr}{\mathbb{R}}
\newcommand{\Cc}{\mathbb{C}}
\newcommand{\Kk}{\mathbb{K}}
\newcommand{\Hh}{\mathbb{H}}


%---- Modifications de symboles -----
\renewcommand {\epsilon}{\varepsilon}
\renewcommand {\le}{\leqslant}
\renewcommand {\ge}{\geqslant}
\renewcommand {\leq}{\leqslant}
\renewcommand {\geq}{\geqslant}

\newcommand{\re}[1]{\mathop{\mathrm{Re}\left(#1\right)}\nolimits}
\newcommand{\im}[1]{\mathop{\mathrm{Im}\left(#1\right)}\nolimits}
\newcommand{\eps}{\varepsilon}
\newcommand{\ssi}{\Leftrightarrow}
\newcommand{\implique}{\Rightarrow}
\newcommand{\vers}{\rightarrow}
\newcommand{\donne}{\mapsto}
\newcommand{\Card}{\text{Card\,}}
\def \m {^{-1}}
\def \Sup  {{\mathrm{Sup}}\: }
\newcommand{\ens}[1]{\left\{ {#1} \right\}}

\newcommand{\Par}[1]{\left({#1}\right)}
\newcommand{\pgcd}{\text{pgcd}}
\newcommand{\ppcm}{\text{ppcm}}
%\newcommand{\val}{\text{val}}
\newcommand{\id}{\text{id}}
\newcommand{\tr}{\text{tr}}

\newcommand{\ch}{\mathop{\mathrm{ch}}\nolimits}
\newcommand{\sh}{\mathop{\mathrm{sh}}\nolimits}
\renewcommand{\tanh}{\mathop{\mathrm{th}}\nolimits}
\renewcommand{\arcsin}{\mathop{\mathrm{arcsin}}\nolimits}
\renewcommand{\arccos}{\mathop{\mathrm{arccos}}\nolimits}
\renewcommand{\arctan}{\mathop{\mathrm{arctan}}\nolimits}
\newcommand{\argsh}{\mathop{\mathrm{argsh}}\nolimits}
\newcommand{\argch}{\mathop{\mathrm{argch}}\nolimits}
\newcommand{\argth}{\mathop{\mathrm{argth}}\nolimits}

\newcommand{\Ker}{\mathop{\mathrm{Ker}}\nolimits}
\newcommand{\rg}{\mathop{\mathrm{rg}}\nolimits}
\newcommand{\Com}{\mathop{\mathrm{Com}}\nolimits}
%\newcommand{\Fr}{\mathop{\mathrm{Fr}}\nolimits}
\newcommand{\diam}{\mathop{\mathrm{diam}}\nolimits}
\newcommand{\absolue}[1]{\left| #1 \right|}
\newcommand{\fonc}[5]{#1 : \begin{cases}#2 \rightarrow #3 \\ #4 \mapsto #5
 \end{cases}}
\newcommand{\transp}[1]{\sideset{^{t}}{}{\mathop{#1}}}


%%%%%%%%%%%%%%%%%%%%%%%%%%%%%%%%%%%%%%%%%%%
% Thermo
%%%%%%%%%%%%%%%%%%%%%%%%%%%%%%%%%%%%%%%%%%
\newcommand{\dpartial}[3]{\ensuremath{\left(\frac{\partial #1}{\partial
        #2}\right)_#3}}
\newcommand{\dpartials}[2]{\ensuremath{\frac{\partial #1}{\partial
        #2}}}
\newcommand{\ddpartial}[3]{\ensuremath{\left(\frac{\partial^2
        #1}{\partial #2^2}\right)_#3}}


%%%%%%%%%%%%%%%%%%%%%%%%%%%%%%%%%%%%%%%%%%%
%�lectrostat
%%%%%%%%%%%%%%%%%%%%%%%%%%%%%%%%%%%%%%%%%%%
%\newcommand{\grad}{\ensuremath{\mathbf{grad}}}


\newcommand{\ud}{\ensuremath{\mrm{d}}}


%\mathindent=2\parindent

\def\d{\mathrm{d}}
%\def\exp{\mathrm{e}}

\let\epsilon\varepsilon
\let\phi\varphi

% \let\Frac\frac
% \def\frac#1#2{\displaystyle\Frac{#1}{#2}}

% \let\Sum\sum
% \def\sum{\displaystyle\Sum}

% \let\Prod\prod
% \def\prod{\displaystyle\Prod}


% \let\Int\int
% \def\int{\displaystyle\Int}

%\def\dbar{\not d}
\let\dbar\delta

\DeclareMathOperator{\grad}{\vect{\mathrm{grad}}}
\DeclareMathOperator{\rot}{\vect{\mathrm{rot}}}
\def\div{\mathop{\operator@font div}\nolimits}

\def\ch{\mathop{\operator@font ch}\nolimits}
\def\sh{\mathop{\operator@font sh}\nolimits}
\def\th{\mathop{\operator@font th}\nolimits}

\def\Re{\mathop{\operator@font Re}\nolimits}
\def\Im{\mathop{\operator@font Im}\nolimits}

\def\Arg{\mathop{\operator@font arg}\nolimits}


\def\oiint{\begingroup\displaystyle\unitlength 1pt\int\mkern-4.5mu
\begin{picture}(0,3)\put(0,3){\oval(10,8)} \end{picture}\mkern-4.5mu
\int\endgroup}

\def\SIm{\@ifstar{}{\,}\mathrm{m}}
\def\SIkm{\@ifstar{}{\,}\mathrm{km}}
\def\SImm{\@ifstar{}{\,}\mathrm{mm}}
\def\SIcm{\@ifstar{}{\,}\mathrm{cm}}
\def\SImicrom{\@ifstar{}{\,}\mu\mathrm{m}}
\def\SInm{\@ifstar{}{\,}\mathrm{nm}}
\def\SIpm{\@ifstar{}{\,}\mathrm{pm}}

\def\SIL{\@ifstar{}{\,}\mathrm{L}}
\def\SImL{\@ifstar{}{\,}\mathrm{mL}}
\def\SIcL{\@ifstar{}{\,}\mathrm{cL}}

\def\SIg{\@ifstar{}{\,}\mathrm{g}}
\def\SIkg{\@ifstar{}{\,}\mathrm{kg}}
\def\SImg{\@ifstar{}{\,}\mathrm{mg}}
\def\SIcg{\@ifstar{}{\,}\mathrm{cg}}

\def\SImolL{\@ifstar{}{\,}\mathrm{mol\cdot L^{-1}}}

\def\SImol{\@ifstar{}{\,}\mathrm{mol}}

\def\SIkgL{\@ifstar{}{\,}\mathrm{kg\cdot L^{-1}}}
\def\SIgL{\@ifstar{}{\,}\mathrm{g\cdot L^{-1}}}
\def\SIkgmcube{\@ifstar{}{\,}\mathrm{kg\cdot m^{-3}}}

\def\SIC{\@ifstar{}{\,}{}^\circ\mathrm C}

\def\SIgmol{\@ifstar{}{\,}\mathrm{g\cdot mol^{-1}}}

\def\SIms{\@ifstar{}{\,}\mathrm{m\cdot s^{-1}}}

\def\SIs{\@ifstar{}{\,}\mathrm{s}}

\def\SIN{\@ifstar{}{\,}\mathrm{N}}
